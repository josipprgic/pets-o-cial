\chapter{Zaključak i budući rad}
		
		Naš zadatak bio je izraditi platformu koja bi vlasnicima kućnih ljubimaca omogućila druženje i komunikaciju s drugim vlasnicima, objavljivanje raznih medijskih sadržaja i pronalazak željenih usluga za njihove ljubimce.
		
		Projekt smo proveli u tri faze:
		\begin{enumerate}
			\item početna razrada funkcionalnih i nefunkcionalnih zahtjeva
			\item implementacija zahtjeva i daljnje razrađivanje
			\item testiranje i završno dokumentiranje
		\end{enumerate} 
	
		U prvoj fazi projekta proučavali smo funkcionalne zahtjeve aplikacije, odnosno što sve korisnici u njoj mogu obavljati. Ova faza bila je dosta teška u početku budući da se kao članovi tima nismo od prije poznavali, pa tako nismo ni previše znali koliko je tko iskusan u kojem području. Također, shvatili smo kako je prije izrade samog projekta potrebno vrlo dobro proučiti alate koje ćemo koristiti. 
		
		U drugoj fazi projekta krenuli smo s implementacijom koja nam je opet bila dosta teška budući da se nismo toliko dobro poznavali. No ipak smo s vremenom počeli implementirati zahtjeve sve lakše. Također, tijekom implementacije smo prepoznali propuste u definicijama funkcionalnih i nefunkcionalnih zahtjeva pa smo ih ponovno i preciznije definirali. 
		
		U trećoj smo fazi, nakon implementacije, dovršili i dokumentaciju projekta i obavili testiranje sustava. Tijekom pisanja dokumentacije smo prepoznali veličinu našeg projekta i vrijeme i organiziranost potrebno u izradi “pravih” projekata. 
		
		Članovi tima su prije projekta bili upoznati s Javom, Reactom i PostgreSQL-om pa smo tako i odabrali tehnologije koje smo koristili. 
		
		Naučili smo važnost dobre koordinacije i komunikacije s članovima tima. Također, naučili smo neke vještine povezane s projektima općenito, poput korištenja alata git i GitLab-a, te izrade dijagrama alatom Astah Professional.
		
		Najveću važnost tijekom projekta možemo pridodati vremenskoj organiziranosti. Ovakvi projekti su vremenski zahtjevni te je potrebna dobra usklađenost da nebi došlo do situacije u kojoj jedna implementacija ovisi o drugoj koja još nije dovršena. Budući da nismo radili na previše ovakvih projekata imali smo i takvih slučajeva, no uspješno smo ih riješili. 
		
		Od definiranih funkcionalnih zahtjeva, implementirali smo prijavu i registraciju korisnika, stvaranje i uređivanje profila ljubimca i korisnika, objavu medijskog sadržaja, komentiranje, slanje zahtjeva za prijateljstvo, stvaranje događaja, slanje poruka, prijava drugih korisnika, blokiranje i brisanje korisnika od strane administratora.  
		
		Tijekom testiranja smo uvidjeli da od početka treba biti temeljit i detaljan pri definiraju slučajeva korištenja i prepoznati rubne slučajeve. 
		
		Zaključno, stekli smo nova iskustva, znanja i susreli se s nekim novim alatima, no najvažnije je da smo dobili dojam kako je to raditi u timu na projektu koji zahtjeva izrazitu vremensku organiziranost i znanje da bi se dobro napravio. Vjerujem da bi projekt, ukoliko počinjemo ispočetka, napravili bolje i brže. 
		
		\eject 